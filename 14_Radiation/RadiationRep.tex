%!TEX root = ../main.tex
%
% 放射線の測定
% レポート用紙
%

\stepcounter{section}
\section*{放射線の測定}

\begin{center}
\begin{tabular}{|c|c|c|c|}
\hline
\parbox[c][1.2cm][c]{0cm}{}学籍番号 & \hspace{3cm} & 名前 & \hspace{6cm} \\
\hline
\parbox[c][1.2cm][c]{0cm}{}実験日時 & \multicolumn{3}{|l|}{   年  月  日  曜日  時限}\\
\hline
\parbox[c][2.0cm][c]{0cm}{}共同実験者 & \multicolumn{3}{|l|}{}\\
\hline
\end{tabular}
\end{center}

\subsection*{実験の概要と目的}

\newpage

\subsection*{測定値および計算}

\subjikken{放射線源試料のエネルギースペクトルの測定}

\hspace*{-\parindent}
\begin{tabular}{|c|c|c|}
\hline
ピークNo. & ピークのエネルギー[keV] & 予想されるγ線源核種 \\
\hline
\hline
&&\\
\hline
&&\\
\hline
&&\\
\hline
&&\\
\hline
&&\\
\hline
&&\\
\hline
&&\\
\hline
&&\\
\hline
&&\\
\hline
&&\\
\hline
&&\\
\hline
&&\\
\hline
&&\\
\hline
&&\\
\hline
&&\\
\hline
\end{tabular}



\subsection*{考察}
\begin{itemize}
\item 線源が有るときと無い時のエネルギースペクトルを比較し、その違いについて考察しましょう。

\vspace{3cm}

\item 我々の身の回りにある放射線源について調べ、その影響について考えてみましょう。

\end{itemize}
