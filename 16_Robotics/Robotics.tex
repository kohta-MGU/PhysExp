%!TEX root = ../main_wo_rep.tex
%
% ロボティクス
%


\section{ロボティクス}

\subsection{はじめに}

現代社会では様々な分野でロボットが活躍し、我々の生活を支えています。ロボットとはセンサーからの情報を電子頭脳(コンピュータ)で分析し、ある程度自立的な作業を機械的に行う装置のことです。そのロボットを支える技術には今まで学んだ物理学の知識が数多く利用されています。例えば、センサーは様々な物理現象(音、光、圧力など)を電気信号としてとらえ、コンピュータはその情報を半導体などを用いた電子回路によって分析します。また、ロボットが動作するための電気・機械的機構(モーターやギアなど)にも電磁気学や力学の原理が必要不可欠です。

この実験ではこの様なロボット工学の原理を学ぶために開発されたレゴ社のマインドストーム NXTを用いて、ロボットがどのように外部の情報をコンピュータのプログラムによって処理しモーターなどの電気・機械部品を制御するのかについて学びます。


\subsection{ロボットの動作}

ロボットとは人間の代わりにある程度自立して何らかの作業を行う装置を指します。そのロボットの動作のためには、人間が普段行っているのと同じように、外部の情報を受け取り、その情報をもとに判断を下し、手足などを使って実際の動作や作業を行う必要があります。

外部の情報を得るための装置は人間の目や耳や触覚に相当し、センサーと呼ばれます。センサーには人間の五感に相当する情報を得るものもありますし、人間の感覚では得ることができない情報をとらえることもできます。センサーは化学反応や電気的な物理現象を利用して外部の情報を電気的な信号に変換します。教材にはタッチセンサー、サウンドセンサー、光センサー、超音波センサーが含まれおり、以下にそれぞれのセンサーの仕組みと役割を述べます。

\begin{itemize}
\item {\bf タッチセンサー}\\
中に機械的なスイッチが入っており、スイッチが押されたことを電流のON/OFFで検知します。

\item {\bf サウンドセンサー}\\
中にはマイクロフォンが内蔵されていて、外部の音の大きさを電気信号として測定します。このセンサーで測定できるのは音の大きさだけで、音の振動数や音色(波形)までは測定することができません。

\item {\bf 光センサー}\\
中には発光ダイオードとフォトダイオード(\S\ref{light sensor}参照)が内蔵されていて、発光ダイオードの光を他の物体に照射し、その反射光の強さをフォトダイオードで測定します。物体の反射率の違いによって返ってくる光の量が変化しますので、それによって物体を識別することができます。このセンサーでは色までを判別することはできませんが、発光ダイオードが赤色の単色の光を発しているので、色による反射率の違いである程度の色の違いは区別することができます。また、フォトダイオードのみを用いることで、周り(環境光)の明るさを知ることもできます。

\item {\bf 超音波センサー}\\
このセンサーには二つの窓が開いており、それぞれ人間の耳には聞こえない高い振動数の音(超音波)の発信機(スピーカ)と受信機(マイクロフォン)になっています。発信機によって発せられた超音波が物体に反射して戻り、受信機で受け取るまでの時間を測定すれば物体までの距離を測定することができます。したがって、この超音波センサーは距離センサーとして働き、$\pm 3$ [cm]の精度で0〜2.5 [m]の距離を測定することができます。


\end{itemize}

それぞれのセンサーはNXTブロックと呼ばれる装置の1番から4番までのポートにケーブルで接続されます。NXTブロックにはマイクロコンピュータが内蔵されており、センサーからの情報をこのコンピュータで処理します。NXTブロックは単体でも簡単なプログラムを組むことができますが、USBポートでパソコンと接続すると、より高度なプログラムを作成することもできます。

センサーによって得られた情報はロボットの頭脳にあたるNXTブロックで処理し、次にどのような動作を行うべきか判断します。その判断に基づいてモーターを制御しロボットの動作を決定します。動作を与えるために出力するポートはNXTブロックのA, B, Cポートです。この出力ポートにモーターなどをケーブルで接続します。

実験ではこれらのセンサー、NXTブロック、モーターを組み合わせ目的の動作を行うロボットを制作します。


\subsection{半導体と光センサー}
\label{light sensor}

\begin{itemize}

\item 半導体\\
シリコンにリンなどを少量混ぜて作った半導体を、N型半導体と呼び、自由電子を多く含んでいます。一方、シリコンにホウ素などを少量混ぜて作った半導体をP型半導体と呼び、自由ホール(電子の欠損した穴)を多く含んでいます。

P型半導体とN型半導体を一個ずつ結合させたものをダイオードと呼び、三個をNPN、又はPNPの組み合わせで結合させたものをトランジスタと呼びます。

\item 発光ダイオード\\
ダイオードにP型がプラス、N型がマイナスになるように電圧を加えると、P型内の自由ホールがN型(マイナス極)の方に、N型内の自由電子がP型(プラス極)の方に移動するので、ダイオードの中央で自由ホールと自由電子が対消滅して光を出します。

\item フォトダイオード\\
発光ダイオードとは逆に、ダイオードのP型にマイナス、N型にプラスの電圧をかけておきます。このままでは電流は流れませんが、光がダイオードに当たって自由電子と自由ホールの対を作ると、これが各々N型(プラス極)とP型(マイナス極)の方に移動するので、電流が流れます。
\end{itemize}

\begin{itembox}[l]{\bf コラム:センサーのはなし}
実験に使う光センサー(フォトダイオード)の他に、身の回りには、温度センサー(エアコン)、湿度センサー(電子レンジ)、ガスセンサー(ガス漏れ検出器)、煙センサー(火災報知器)、磁気センサー(オーディオ製品)、圧力センサー(水位チェック)など、様々なセンサーが開発、利用されています。光センサーには、可視光線以外の光を利用する赤外線センサー(気象衛星)などもあります。
\end{itembox}







\newpage

\jikken

\begin{itemsquarebox}[c]{\bf 実験用具}
教育用レゴ マインドストーム NXT、パソコン
\end{itemsquarebox}

\bigskip

\subjikken{各センサーの働きを調べる}

\begin{enumerate}

\item タッチセンサー、サウンドセンサー、光センサー、超音波センサーをそれぞれNXTブロックの1番から4番までのポートに接続します。

\item ランプをNXTブロックのポートA、2つのモーターユニットをポートB, Cにそれぞれ接続します。

\item デモプログラムを動作させます。\\
{\tt MyFiles $>$ Software files $>$ DemoV2 $>$ DemoV2 Run}

\item 各センサーから得られる情報を数値として確認します。\\
{\tt View $>$ Touch/Sound dB/Reflected light/Ambient light/Ultrasonic cm$>$ Port 1$\sim$4}
(ポートは各センサーが接続されているポート番号を指定します。)\\
特に、いろいろな物体(色)の反射光に対する光センサーの値、および超音波センサーが示す距離と巻尺で測った実際の距離とのずれを測定し、記録します。また、他にもモーターからの情報なども得ることができますが、それらの機能の意味するところを考えてみましょう。

\item 各センサーからの情報をもとに制御されているモーターの動作を確認する。\\
{\tt Try Me $>$ Try-Touch/Sound/Light/Ultrasonic/Motor $>$ Run}\\
どのようなプログラムを作成すればこのような動作が可能になるか、考えてみましょう。




\end{enumerate}

\subjikken{基本ロボットの組み立て}

\begin{enumerate}

\item 組み立てガイドに従って基本のロボットを組み立てます。ロボットの右タイヤ部分(8〜12ページ)、左タイヤ部分(13〜15ページ)、中央のキャスター(小タイヤ)部分(16〜17ページ)は独立に組み立てができるので、人数が多いグループは分担して組み立てると良いでしょう。

\item ケーブルなどの接続を行いロボットの基本部分が完成したら、組み立てガイド23ページにあるプログラムをNXTブロックに直接入力(NXT Program)して動作させます。 正しく動作するか確認し、入力したプログラムと実際の動作を比較し、プログラムの意味を考察しましょう。

\item NXTブロックの使い方や、パソコンでプログラミングするソフトウェアの使い方などについても習熟し、自分たちで考えた他のプログラムも動作させてみましょう。

\end{enumerate}

\subjikken{光センサーの取り付けとライントレーサ}

\begin{enumerate}

\item 組み立てガイドの32〜34ページに従って光センサーを取り付けます。

\item 組み立てガイドの35ページにあるプログラムをNXTブロックに入力し、ライントレーサをプログラムします。ロボットは黒と白の境界に沿って正しく走るでしょうか? 今入力したプログラムでなぜ境界に沿ってロボットが走行するのか考えてみましょう。

\item パソコンのプログラミングツールにある\\
「共通パレット $\gg$ 17. ライン上を進む」\\
のプログラミングガイドに従ってライントレーサのプログラムを作成しましょう。作成したプログラムでロボットが正しく動作することを確認し、作成したプログラムの動作原理について考察しましょう。

\item 例題のプログラムを改良し、より速く正確にロボットがライン上を走行するようにしましょう。プログラムだけではなくロボットの走行機構部分にも工夫をこらし、コース上のラインを一周する時間ができるだけ短くなるように挑戦してみましょう。

\item ロボットのセンサーと機械部分はそのままにしておいて、プログラムだけを変更することで、他にどのような動きをするロボットを作成できるか考えましょう。(例:ライトレース用のコースの線の内側から出ることなく走行するロボット。)


\end{enumerate}
