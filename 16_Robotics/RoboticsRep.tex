%!TEX root = ../reports.tex
%
% ロボティクス
% レポート用紙
%

\stepcounter{section}
\section*{ロボティクス}

\begin{center}
\begin{tabular}{|c|c|c|c|}
\hline
\parbox[c][1.2cm][c]{0cm}{}学籍番号 & \hspace{3cm} & 名前 & \hspace{6cm} \\
\hline
\parbox[c][1.2cm][c]{0cm}{}実験日時 & \multicolumn{3}{|l|}{   年  月  日  曜日  時限}\\
\hline
\parbox[c][2.0cm][c]{0cm}{}共同実験者 & \multicolumn{3}{|l|}{}\\
\hline
\end{tabular}
\end{center}

\subsection*{実験の目的}

\vspace{8cm}

\subsection*{実験結果}

\subjikken{}

\subsubsection*{光センサーの測定}

\begin{tabular}{|c|c|}
\hline
材質および色 & 数値\\
\hline\hline
\hspace*{3cm}&\hspace*{3cm}\\
\hline
&\\
\hline
&\\
\hline
&\\
\hline
&\\
\hline
\end{tabular}

\subsubsection*{超音波センサーの測定}

\begin{tabular}{|c|c|c|}
\hline
センサーの測定値 & 巻尺で測った距離 & ずれ\\
\hline\hline
\hspace*{3cm}&\hspace*{3cm}&\hspace*{3cm}\\
\hline
&&\\
\hline
&&\\
\hline
&&\\
\hline
&&\\
\hline
\end{tabular}


\bigskip\bigskip

\subjikken

\subsubsection*{NXTブロックに直接入力した組み立てガイド23ページにあるプログラムの意味と実際の動作}

\vspace{7cm}

\subjikken

\subsubsection*{NXTブロックに直接入力したライントレーサプログラム(組み立てガイドの35ページ)の意味と実際の動作}

\newpage

\subsubsection*{パソコンのNXTプログラミングで入力したライントレーサプログラムの動作原理およびフローチャート}

\vspace{7cm}

\subsubsection*{ライントレーサを改良したプログラムのフローチャートおよび工夫した点(どの程度時間が短縮できたか記載すること)}

\vspace{7cm}

\subsubsection*{ライントレーサ以外のプログラムを組んだ場合、その動作原理および工夫した点}

\newpage

\subsection*{考察}

\begin{itemize}

\item ロボットの動作を完全に機械的に制御する場合(からくり人形など)とプログラムで制御する場合を比較して、その特徴(利点・欠点)について考察しましょう。

\vspace{6cm}

\item ロボットの動作をより正確なものに効率よく近づけていくためには、どのような作業や工夫が必要か考えてみましょう。

\vspace{6cm}

\item 今回の実験で用いたセンサーや動力機構を使って、他にどのようなロボットを製作できるか、また我々の生活にどのように役立てることができるか、考えてみましょう。

\end{itemize}

