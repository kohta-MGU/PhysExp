%!TEX root = ../main.tex
%
% 固体の比熱
%


\section{固体の比熱}

\subsection{比熱とは}

ある物質1 [g]の温度を1度上げるのに必要な熱量(エネルギー)を、その物質の比熱と呼びます。 
例えば、質量が$m$ [g]の物質があるとして、この物質の温度を$T$度上げるのに必要な熱 
量が$Q$ [J] (J:ジュール)であったとすると、この物質の比熱は次のように表さ 
れます。
\[
C=\frac{Q}{m T} \quad \text{[J/gK]}
\]

ここでは、混合法と呼ばれる方法を用いて固体の比熱を求めてみましょう。混合法 
とは、すでに比熱の分かっている水などを使い、固体から水に移る熱量を調べること 
で、目的となる固体の比熱を測定する方法のことを言います。


\subsection{混合法の測定原理}

100℃に熱した固体を水に入れると、固体の温度は下がり、水は固体の熱を受け取 
って温度が上昇します。最終的には水と固体は同じ温度になりますが、このとき、熱 
が外部に逃げ出していなければ、水の受け取った熱量は固体が失った熱量と一致しま 
す。

例えば、質量$m$ [g]、比熱$C$ [J/gK]の固体があったとします。この固体を$T_1$ [℃]に 
熱したものを、質量が$M$ [g]、温度が$T_2$ [℃]の水(比熱を$C_W=4.2$ [J/gK]とする)に入れ、固体と 
水の温度が一緒になるまで撹拌した後、水の温度を測定したところ、$T_3$ [℃]になって 
いたとします。このとき、固体が失った熱量は、
\begin{equation}
m C (T_1-T_3)
\label{solid}
\end{equation}
となります。また、水が得た熱量は、
\begin{equation}
M C_W (T_3-T_2)
\label{water}
\end{equation}
となります。熱が外部に逃げなかった場合、(\ref{solid})と(\ref{water})の熱量は等しくなるため、 
次の式が成り立ちます。
\begin{equation}
m C (T_1-T_3) = M C_W (T_3-T_2)
\label{equilibrium}
\end{equation}

この式から固体の比熱$C$を求めてやることができます。ただし、実際の実験では、
水を入れる容器と撹拌する棒の影響を考える必要があるので、(\ref{equilibrium})式の右辺に補正 
を加える必要があります。

水を入れる容器と撹拌棒(容器と同じ材質とする)の質量の合計を$M'$、比熱を 
$C'$(材質が銅の場合は0.38 [J/gK])として(\ref{equilibrium})の式に補正を加えてやると、
\begin{equation}
m C (T_1-T_3) = (M C_W+M'C') (T_3-T_2)
\end{equation}
となります。この式から、固体の比熱$C$は、次のように求められます。
\begin{equation}
\boxed{
C = \frac{(M C_W+M'C') (T_3-T_2)}{m (T_1-T_3) }
}
\label{calibrated}
\end{equation}


\bigskip

\begin{itembox}[l]{\bf コラム:気体の比熱}
気体を温めると、一般に体積が膨張するので、体積を一定に保ちながら温度を上げる場合 
の比熱(定積比熱:$C_V$)と、圧力を一定に保ちつつ温度を上げる場合の比熱(定圧比熱:$C_p$) 
は、異なる大きさになる。理想気体では、2種類の比熱の間に
\[
C_p-C_v=R 
\]
という関係がある。ここで、$R$は1モルの気体定数で$R=8.314472$ [J/mol/K]である。機 
密性のよいマンションの部屋を暖房するときに必要な熱量を計算するときは定積比熱
を使う。一方、隙間だらけの和室の暖房に必要な熱量を計算するときはこのどちらも使
うことが出来ない。温度を上げると、和室の体積も圧力も変わらないが、温められて膨
張した空気が隙間から外に逃げていくからである。
\end{itembox}



\newpage

\jikken

\begin{itemsquarebox}[c]{\bf 実験用具}
恒温水槽、温度計、水熱量計(銅製の容器と撹拌棒)、比熱測定用試料(鉄、アルミニウム)、デジタルはかり、ストップウォッチ
\end{itemsquarebox}

\bigskip

\subjikken{混合法による固体の比熱の測定}

1つの試料について、2度測定を繰り返し、その平均を求めること。また、実験 
は素早く行なう必要があるため、1, 2回予備実験で手順を練習し、それから測 
定に入った方が良い。

\begin{enumerate}

\item 恒温水槽に試料が完全に浸かる量の水を入れ、最高温度(80℃)に設定して加熱 
を始めておきます。

\item 水熱量計の銅製の容器と銅製の撹拌棒の質量の和$M'$を測定し、試料の重さ$m$も 
測定しておきます。(撹拌棒のつまみの部分は外して測定すること。)

\item 容器をはかりに置いた状態で電源をONにして表示がゼロになるようにセットし、 
容器に半分より少し多い程度(約100g)の水を入れます。この状態ではかりの 
表示を読み、入れた水の質量$M$を求めます。その後、容器の蓋と撹拌棒のつま 
みを取り付け、容器に温度計をセットし、最初の水温$T_2$を測定しておきます。

\item 恒温水槽が最高温度$T_1$に達したら、試料をお湯の中に3分間吊るして入れてお 
きます。(このとき、試料がヒーターに接触しないように注意すること。)

\item 試料をお湯から出し、冷めないうちに、すばやく水熱量計の容器の水の中に入れ、 
ストップウォッチで計測を始めます。その後、撹拌棒で水をかきまぜながら、10〜15秒毎に水温を記録します。
(水温の上昇が止まるまで続けること。)水温の上昇が 
止まった時点での水の温度が$T_3$の値になります。

\item 前のページの(\ref{calibrated})式を使って、試料の比熱の値を導出しましょう。(但し、銅の 
比熱は0.38 [J/gK]として計算すること。)

\end{enumerate}



