%!TEX root = ../main.tex
%
% ニュートンリング
% レポート用紙
%

\stepcounter{section}
\section*{ニュートンリング}

\begin{center}
\begin{tabular}{|c|c|c|c|}
\hline
\parbox[c][1.2cm][c]{0cm}{}学籍番号 & \hspace{3cm} & 名前 & \hspace{6cm} \\
\hline
\parbox[c][1.2cm][c]{0cm}{}実験日時 & \multicolumn{3}{|l|}{   年  月  日  曜日  時限}\\
\hline
\parbox[c][2.0cm][c]{0cm}{}共同実験者 & \multicolumn{3}{|l|}{}\\
\hline
\end{tabular}
\end{center}

\subsection*{実験の目的}

\vspace{5cm}


\subsection*{測定値と計算結果}

\subjikken{}
\subsubsection*{暗線の位置と半径}

\hspace*{-\parindent}
\begin{tabular}{|p{4.5cm}|p{4.5cm}|p{6cm}|}
\hline
左の輪の位置 [mm] & 右の輪の位置 [mm] & 半径 [mm] ($r_n=\frac{a_n'-a_n}{2}$)\\
\hline\hline
$a_{10}=$&$a'_{10}=$&$r_{10}=$\\
\hline
$a_9=$&$a'_{9}=$&$r_{9}=$\\
\hline
$a_8=$&$a'_{8}=$&$r_{8}=$\\
\hline
$a_7=$&$a'_{7}=$&$r_{7}=$\\
\hline
$a_6=$&$a'_{6}=$&$r_{6}=$\\
\hline
$a_5=$&$a'_{5}=$&$r_{5}=$\\
\hline
\end{tabular}

\newpage

\subsubsection*{波長の計算}
\begin{eqnarray}
\lambda(m=5) &=& \frac{r_8^2-r_5^2}{6000} = \hspace{5cm} \text{[mm]}\nonumber\\
\lambda(m=6) &=& \frac{r_9^2-r_6^2}{6000} =  \hspace{5cm} \text{[mm]}\nonumber\\
\lambda(m=7) &=& \frac{r_{10}^2-r_7^2}{6000} =\hspace{5cm} \text{[mm]}\nonumber
\end{eqnarray}

\subsubsection*{平均値}

\begin{equation}
\lambda(平均)= \hspace{8cm}\text{[mm]}\nonumber
\end{equation}


\subsection*{考察}

\begin{itemize}

\item 理科年表で調べたナトリウムランプの波長と実験結果を比較しましょう。

\end{itemize}

