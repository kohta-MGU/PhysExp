%!TEX root = ../main.tex
%
% 電流と磁場
% レポート用紙
%

\stepcounter{section}
\section*{電流と磁場}

\begin{center}
\begin{tabular}{|c|c|c|c|}
\hline
\parbox[c][1.2cm][c]{0cm}{}学籍番号 & \hspace{3cm} & 名前 & \hspace{6cm} \\
\hline
\parbox[c][1.2cm][c]{0cm}{}実験日時 & \multicolumn{3}{|l|}{   年  月  日  曜日  時限}\\
\hline
\parbox[c][2.0cm][c]{0cm}{}共同実験者 & \multicolumn{3}{|l|}{}\\
\hline
\end{tabular}
\end{center}

\subsection*{実験の目的}

\vspace{5cm}

\subsection*{実験の記録}

\subjikken{作成した簡易スピーカーのスケッチ}

\newpage

\subjikken{作成した簡易モーターのスケッチ}

\vspace{7cm}


\subsection*{考察}

\begin{itemize}

\item 作成した簡易スピーカーでなぜ音が出るのでしょうか? 原理について考察して見ましょう。

\vspace{5cm}

\item 簡易スピーカーはマイクとしても働きます。なぜでしょうか?

\newpage

\item 簡易モーターの作成で、片方のエナメル線を半分だけ剥がすのはなぜでしょうか? モーターの原理を含めて考察してみましょう。

\vspace{6cm}

\item 今回の実験で見たような電流と磁場の関係(性質)を用いた身の回りの応用例について調べてみましょう。

\end{itemize}

