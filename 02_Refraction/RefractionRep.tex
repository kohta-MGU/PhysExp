%!TEX root = ../reports.tex
%
% 光の性質
% レポート用紙
%

\stepcounter{section}
\section*{光の性質}

\begin{center}
\begin{tabular}{|c|c|c|c|}
\hline
\parbox[c][1.2cm][c]{0cm}{}学籍番号 & \hspace{3cm} & 名前 & \hspace{6cm} \\
\hline
\parbox[c][1.2cm][c]{0cm}{}実験日時 & \multicolumn{3}{|l|}{   年  月  日  曜日  時限}\\
\hline
\parbox[c][2.0cm][c]{0cm}{}共同実験者 & \multicolumn{3}{|l|}{}\\
\hline
\end{tabular}
\end{center}

\subsection*{実験の目的}

\vspace{7cm}


\subsection*{観察、測定、および計算}
\subjikken{プリズムによる分光(スケッチ)}
\newpage


\subjikken{レーザー光とプリズム(スケッチ)}
\vspace{5.5cm}

\subjikken{光の屈折と反射(スケッチ)}
\vspace{8.5cm}

\subjikken{光ファイバー(スケッチ)}


\newpage

\subjikken{臨界角}
\subsubsection*{光学水槽で求めた臨界角}
\begin{equation}
\theta = \hspace{5cm}度\nonumber
\end{equation}
\subsubsection*{水の屈折率}
\begin{equation}
n_{水} =\frac{1}{\sin\theta}=\hspace{5cm}\nonumber
\end{equation}

\bigskip\bigskip

\subjikken{糖度計}
\subsubsection*{資料の糖度の測定}
\hspace*{-\parindent}
\begin{tabular}{|p{4cm}|p{4cm}|}
\hline
資料 & 糖度 [Brix\%] \\
\hline\hline
&\\
\hline
&\\
\hline
&\\
\hline
&\\
\hline
\end{tabular}




\subsection*{考察}

\begin{itemize}

\item 太陽光(白色光)とレーザー光をそれぞれプリズムに入射した時の違いについて説明し、なぜその違いが生じるのかを考察しましょう。

\newpage

\item 今回行った実験から光ファイバーの原理について考察しましょう。

\vspace{7cm}

\item 今回実験した光の性質(反射、屈折、全反射、偏光)は我々の身の回りでどのような現象として見られるでしょうか? また、どのようなものに利用されているでしょうか?  色々な例を調べてみましょう。


\end{itemize}

