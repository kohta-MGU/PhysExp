%!TEX root = ../main.tex
%
% 光の干渉
% レポート用紙
%

\stepcounter{section}
\section*{光の干渉}

\begin{center}
\begin{tabular}{|c|c|c|c|}
\hline
\parbox[c][1.2cm][c]{0cm}{}学籍番号 & \hspace{3cm} & 名前 & \hspace{6cm} \\
\hline
\parbox[c][1.2cm][c]{0cm}{}実験日時 & \multicolumn{3}{|l|}{   年  月  日  曜日  時限}\\
\hline
\parbox[c][2.0cm][c]{0cm}{}共同実験者 & \multicolumn{3}{|l|}{}\\
\hline
\end{tabular}
\end{center}

\subsection*{実験の目的}

\vspace{5cm}


\subsection*{測定値と計算結果}

\setcounter{exp}{0}

\subjikken{}
\subsubsection*{測定値と計算}
\hspace*{-\parindent}
\begin{tabular}{|p{1.5cm}|p{3.2cm}|p{3.2cm}|p{3.2cm}|p{3.2cm}|}
\hline
光源の色 & スリットの間隔 $D$ [mm] & スリット板からスクリーンまでの距離 $L$ [mm] & 明点と中心の距離(干渉縞の間隔) $x$ [mm] & 波長 %$\lambda=\frac{xD}{L}$
 [nm] \\
\hline\hline
&&&&\\
\hline
&&&&\\
\hline
&&&&\\
\hline
&&&&\\
\hline
&&&&\\
\hline
&&&&\\
\hline
&&&&\\
\hline
&&&&\\
\hline
\end{tabular}

\newpage

\subjikken{}
\subsubsection*{測定値と計算}
\hspace*{-\parindent}
\begin{tabular}{|p{1.5cm}|p{3.2cm}|p{3.2cm}|p{3.2cm}|p{3.2cm}|}
\hline
光源の色 & 回折格子の間隔 $d$ [mm] & 回折格子からスクリーンまでの距離 $L$ [mm] & 明点の間隔   $x$ [mm] & 波長
%$\lambda=\frac{xd}{L}$
[nm] \\
\hline\hline
&&&&\\
\hline
&&&&\\
\hline
&&&&\\
\hline
&&&&\\
\hline
&&&&\\
\hline
&&&&\\
\hline
&&&&\\
\hline
&&&&\\
\hline
\end{tabular}

\bigskip\bigskip\bigskip\bigskip

\subjikken{}
\subsubsection*{測定値と計算}
\hspace*{-\parindent}
\begin{tabular}{|p{1.5cm}|p{3.2cm}|p{3.2cm}|p{3.2cm}|p{3.2cm}|}
\hline
光源の色 & 波長 $\lambda$ [nm] & CDからスクリーンまでの距離 $L$ [mm] & 明点の間隔   $x$ [mm] & トラックピッチ
% $d=\frac{\lambda L}{x}$
[$\mu$m] \\
\hline\hline
&&&&\\
\hline
&&&&\\
\hline
&&&&\\
\hline
&&&&\\
\hline
\end{tabular}

\newpage

\subsection*{考察}

\begin{itemize}

\item ダブルスリットの場合と回折格子の場合で、どちらの方がより正確に光の波長が求められたでしょうか? その結果と理由について、
測定時の状況も踏まえて考察してみましょう。

\vspace{6cm}

\item 光の干渉の性質を使って、CDのトラックピッチのような小さいものがなぜ測定できるのか、考察してみましょう。

\vspace{6cm}

\item 今回の実験で学んだ光の干渉の性質は身の回りの現象として見ることができるでしょうか? また、どのようなものに利用されているでしょうか? 例を調べてみましょう。

\end{itemize}

