%!TEX root = ../main.tex
%
% 光の干渉
% レポート用紙
%

\stepcounter{section}
\section*{光の干渉}

\begin{center}
\begin{tabular}{|c|c|c|c|}
\hline
\parbox[c][1.2cm][c]{0cm}{}学籍番号 & \hspace{3cm} & 名前 & \hspace{6cm} \\
\hline
\parbox[c][1.2cm][c]{0cm}{}実験日時 & \multicolumn{3}{|l|}{   年  月  日  曜日  時限}\\
\hline
\parbox[c][2.0cm][c]{0cm}{}共同実験者 & \multicolumn{3}{|l|}{}\\
\hline
\end{tabular}
\end{center}

\subsection*{実験の目的}

\vspace{5cm}


\subsection*{測定値と計算結果}

\subjikken{}
\subsubsection*{スリットの幅}
\begin{equation}
d = \hspace{5cm}\text{[mm]}\nonumber
\end{equation}
\subsubsection*{スリット板からスクリーンまでの距離}
\begin{equation}
L = \hspace{5cm}\text{[mm]}\nonumber
\end{equation}
\subsubsection*{最初の暗点と中心の明点までの距離}
\begin{equation}
x= \hspace{5cm}\text{[mm]}\nonumber
\end{equation}
\subsubsection*{レーザー光の波長}
\begin{equation}
\lambda= \frac{d x}{L}=\hspace{5cm}\text{[nm]}\nonumber
\end{equation}

\bigskip

\subjikken{}
\subsubsection*{測定値と計算}
\hspace*{-\parindent}
\begin{tabular}{|p{1.5cm}|p{3.2cm}|p{3.2cm}|p{3.2cm}|p{3.2cm}|}
\hline
光の色 & スリットの間隔 $D$ [mm] & スリット板からスクリーンまでの距離 $L$ [mm] & 明点と中心の距離 $x$ [mm] & 波長 $\lambda=\frac{xD}{L}$ [nm] \\
\hline\hline
&&&&\\
\hline
&&&&\\
\hline
&&&&\\
\hline
&&&&\\
\hline
\end{tabular}

\bigskip\bigskip\bigskip\bigskip

\subjikken{}
\subsubsection*{測定値と計算}
\hspace*{-\parindent}
\begin{tabular}{|p{1.5cm}|p{3.2cm}|p{3.2cm}|p{3.2cm}|p{3.2cm}|}
\hline
光の色 & 回折格子の間隔 $D$ [mm] & 回折格子からスクリーンまでの距離 $L$ [mm] & 明点と中心の距離 $x$ [mm] & 波長 $\lambda=\frac{xD}{L}$ [nm] \\
\hline\hline
&&&&\\
\hline
&&&&\\
\hline
&&&&\\
\hline
&&&&\\
\hline
\end{tabular}


\newpage

\subsection*{考察および感想}


