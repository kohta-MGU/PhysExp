%!TEX root = ../main.tex
%
% オシロスコープ
% レポート用紙
%

\stepcounter{section}
\section*{オシロスコープ}

\begin{center}
\begin{tabular}{|c|c|c|c|}
\hline
\parbox[c][1.2cm][c]{0cm}{}学籍番号 & \hspace{3cm} & 名前 & \hspace{6cm} \\
\hline
\parbox[c][1.2cm][c]{0cm}{}実験日時 & \multicolumn{3}{|l|}{   年  月  日  曜日  時限}\\
\hline
\parbox[c][2.0cm][c]{0cm}{}共同実験者 & \multicolumn{3}{|l|}{}\\
\hline
\end{tabular}
\end{center}

\subsection*{実験の概要と目的}

\vspace{6cm}

\subsection*{測定装置(オシロスコープ)の様子(スケッチ)と簡単な使い方}


\newpage


\subsection*{観察、測定、および計算}

\subjikken{波形の観察と波形から求めた周波数}

\begin{enumerate}

\item 発信器からの正弦波

\vspace{10cm}

\item 音叉

\newpage

\item 自分の声

\vspace{10cm}

\item うなり

\end{enumerate}

\newpage

\subjikken{リサジュー図形の観察とそれを用いた音叉の周波数の測定}

\vspace{7cm}


\subsection*{考察}

\begin{itemize}

\item 自分の声の波形の観察から周波数以外にわかることはあるでしょうか?

\vspace{5cm}

\item オシロスコープを使う利点はなんでしょうか? またオシロスコープはどのような所で使われているでしょうか? 調べてみましょう。

\end{itemize}

