%!TEX root = ../main_wo_rep.tex
%
% 超伝導
%


\section{超伝導}

\subsection{超伝導とは }

ある特定の金属や化合物を非常に低い温度まで冷却していくと、ある温度を境に電気抵抗が急激にゼロに低下する現象があり、
この現象のことを{\bf 超伝導}と呼びます。この現象はオランダの物理学者、カマリング・オネスによって発見され、1913年のノーベル
物理学賞の受賞対象となっています。

もともと電気抵抗(電気伝導性)は物質の温度が低くなると徐々に小さくなって行きますが、超伝導が起きる時には、
温度の下限である絶対零度
(-273.15℃)に達する前に電気抵抗が相転移と呼ばれる現象で突然ゼロになります。カマリング・オネスは当初、水銀を冷却する過程で、
4.19K(-268.96℃)で超伝導状態になることを発見しました。
その後、より高い温度で超伝導状態になる物質も次々に発見されています。
特に、銅酸化物を焼き固めたセラミック系の超伝導物質は、液体窒素程度の温度(77K、-196℃)でも超伝導
状態となります。現在でもより高い温度で超伝導を示す物質やより安価に製造できる超伝導物質の研究が盛んに行われています。
%(現在では-138℃で超伝導を示す物質まで発見されています。)

\subsection{低温の世界}

超伝導を含め-200℃よりも低い低温状態では物質は通常と異なる振る舞いをすることがあります。特に絶対零度0Kに近い温度を
極低温と呼び、極低温状態で発生する様々な物理現象は興味深い研究対象となっています。

低温状態の作り方は、液体窒素や液体ヘリウムなどあらかじめ液化された気体(ガス)で冷却する方法や、断熱消磁法など様々な方法が
あります。
かつては超伝導状態になる数K程度の極低温を作り出すため大型の実験装置が必要でしたが、
最近になって比較的入手が容易な液体窒素(-196℃)の温度であっても超伝導状態を示す物質が発見されたことにより、
超伝導の簡易な実験室での実験や工業的な利用が進みました。

超伝導以外でも、低温の液体窒素を使うと、温度による気体の体積の急激な変化やライデンフロスト現象など、普段あまり見ることが
できない物理現象を見ることができます。

\subsection{マイスナー効果}

電気抵抗がゼロになる現象に加えて、超伝導物質では{\bf  マイスナー効果}と呼ばれる物質内に磁場(磁束)が入り込めない現象が
起こります。超伝導体の中には単に物質内に磁場が入り込めないだけでなく、磁束が細くピン留されて位置が固定される現象も
起こります。このマイスナー効果とピン留効果によって、超伝導状態の物質の上にネオジム磁石など強力な磁石をおくと、
安定した状態で浮上し続けます。この現象を超伝導磁気浮上と呼びます。


\bigskip

\begin{itembox}[l]{\bf コラム:超伝導の利用}

コイル状の超伝導物質を冷却し、超伝導状態のまま電流を流すと、その電気抵抗がゼロとなる性質により発熱など
エネルギーの損失を発生させることなく大電流を流し、強力な電磁石とすることができます。
超伝導物質を用いた超伝導電磁石は、非常に強力な磁場が得られることから、磁気浮上式鉄道(リニアモーターカー)や
MRI(核磁気共鳴画像診断装置)といった医療機器としてすでに実用化され応用が進んでいます。

また、将来的に発電所から各地に電力を送るための送電線に超伝導を利用することができれば、今まで
電気抵抗によって失われていたエネルギーを大幅に減らすことができると考えられています。


\end{itembox}

\newpage

\jikken

\begin{itemsquarebox}[c]{\bf 実験用具}
液体窒素、超伝導物質、ピンセット、手袋、発泡スチロール容器、デュワー瓶、風船、
超伝導線材、直流電源装置、デジタルマルチメーター、パソコン、接続用ケーブル
\end{itemsquarebox}

\bigskip

\subjikken{液体窒素を使った実験}


\begin{enumerate}

\item 液体窒素をデュワー瓶に注ぎます。
\item 空気を少量入れた風船をデュワー瓶の中の液体窒素に入れます。風船の大きさの変化を観察しましょう。
風船を出し、温度が高くなった場合の変化も観察します。
\item 液体窒素をテーブルの上に少量たらし、その様子を観察します。

\end{enumerate}

\bigskip


\subjikken{マイスナー効果}

\begin{enumerate}

\item 2重にした発泡トレイの中に液体窒素を注ぎ、その中でタブレット状の超伝導物質を冷却します。
\item 泡の発生が収まり、超伝導物質が十分冷えたら、その上に小さなネオジム磁石をピンセットで乗せます。
\item 超伝導状態の物質の上の磁石の様子を観察します。(ピンセットで磁石に少し触れてみる。)

\end{enumerate}


\bigskip

\subjikken{電気抵抗率の測定}

\begin{enumerate}

\item  超伝導線材の両端の電線に直流電源装置をつなぎ、電圧調整つまみを全て真ん中、電流調整つまみを全て最小にしておきます。
\item 中央の2本の電線にパソコンと接続したデジタルマルチメータを接続します。
\item パソコンのソフトを使って、端子間の電圧測定のグラフを表示させます。
\item 電流調整つまみをまわして、3A程度の電流を流し、端子間に電圧が生じていることを確認します。(超伝導体の電気抵抗$R$は電流$I$と電圧$V$に対して、
オームの法則により、$R=V/I$の関係が成り立ちます。)
\item 超伝導線材を液体窒素の中に入れ、端子電圧(電気抵抗)がゼロになったことを確認します。
\item 超伝導線材を液体窒素から出し、温度の上昇とともに電圧(電気抵抗)がどのように変化するか確認します。

\end{enumerate}

\bigskip
\hspace*{-\parindent}
※ {\bf 注意}:液体窒素で冷却された超伝導物質・線材やネオジム磁石を素手で絶対に触らない。必ずピンセットや手袋を使用すること。

