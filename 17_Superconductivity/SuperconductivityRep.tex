%!TEX root = ../reports2.tex
%
% 超伝導
% レポート用紙
%

\stepcounter{section}
\section*{超伝導}

\begin{center}
\begin{tabular}{|c|c|c|c|}
\hline
\parbox[c][1.2cm][c]{0cm}{}学籍番号 & \hspace*{3cm} & 名前 & \hspace*{6cm} \\
\hline
\parbox[c][1.2cm][c]{0cm}{}実験日時 & \multicolumn{3}{|l|}{   年  月  日  曜日  時限}\\
\hline
\parbox[c][2.0cm][c]{0cm}{}共同実験者 & \multicolumn{3}{|l|}{}\\
\hline
\end{tabular}
\end{center}

\subsection*{実験の目的}

\vspace{4cm}


\subsection*{実験の記録}

\setcounter{exp}{1}
\subjikken{ネオジム磁石上の超伝導コイルの様子(スケッチ)}

\newpage

\subjikken{超伝導物質の電気抵抗率の測定}
\subsubsection*{温度上昇時の電圧の変化(グラフ)}
\vspace{8cm}

\subsection*{考察}

\begin{itemize}

\item 超低温状態はどのようなものに使われて利用されているでしょうか? 調べてみましょう。

\vspace{4cm}

\item 超伝導状態の物質を使ってどのような利用や応用が期待されているでしょうか? 調べてみましょう。


\end{itemize}
