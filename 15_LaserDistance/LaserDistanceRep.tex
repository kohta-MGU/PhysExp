%!TEX root = ../main.tex
%
% レーザー光による距離測定
% レポート用紙
%

\stepcounter{section}
\section*{レーザー光による距離測定}

\begin{center}
\begin{tabular}{|c|c|c|c|}
\hline
\parbox[c][1.2cm][c]{0cm}{}学籍番号 & \hspace{3cm} & 名前 & \hspace{6cm} \\
\hline
\parbox[c][1.2cm][c]{0cm}{}実験日時 & \multicolumn{3}{|l|}{   年  月  日  曜日  時限}\\
\hline
\parbox[c][2.0cm][c]{0cm}{}共同実験者 & \multicolumn{3}{|l|}{}\\
\hline
\end{tabular}
\end{center}

\subsection*{実験の目的}

\vspace{5cm}

\subsection*{測定値および計算}

\subjikken{}

\subsubsection*{変調波の周波数と周期}
\hspace*{-\parindent}
\begin{tabular}{|c|c||c|c|}
\hline
周波数 [Hz] & 周期 [s] & 周波数 [Hz] & 周期 [s] \\
\hline\hline
\hspace*{3cm}&\hspace*{3cm}&\hspace*{3cm}&\hspace*{3cm}\\
\hline
&&&\\
\hline
&&&\\
\hline
&&&\\
\hline
&&&\\
\hline
&&&\\
\hline
&&&\\
\hline
&&&\\
\hline
&&&\\
\hline
\end{tabular}

\newpage

\subjikken{}

\subsubsection*{誤差校正用測定}

\[
\Delta t_0 = \hspace{3cm}\text{[ns]}
\]

\subsubsection*{距離の計測}

\hspace*{-\parindent}
\begin{tabular}{|c|c|c|c|}
\hline
$\Delta t$ [ns] & 距離 [m] & 巻尺での測定値 [m] & 誤差 [m] \\
\hline\hline
\hspace*{3cm}&\hspace*{3cm}&\hspace*{3cm}&\hspace*{3cm}\\
\hline
&&&\\
\hline
&&&\\
\hline
&&&\\
\hline
&&&\\
\hline
&&&\\
\hline
\end{tabular}


\subsection*{考察}

