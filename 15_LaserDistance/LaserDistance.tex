%!TEX root = ../main.tex
%
% レーザー光による距離測定
%


\section{レーザー光による距離測定}

\subsection{光の速さとは}

光は電磁波であり、真空中を1秒間に299,792,458 [m]伝播します。すなわち、真空中の 
光速は$c = 2.99792458\times 10^8$ [m/s]となります。
この光速$c$の値は、光源や観測者の運動に 
かかわらず、また光(電磁波)の種類(電波、赤外線、可視光線、紫外線、X線、$\gamma$線など)
にかかわらず、同質の媒質では常に同じ値となります。
一方、空気や水などの物質中では、光速は真空中の値より遅くなります。

\subsection{光による距離測定}

ある地点Aとある地点Bの間の距離を測定したい場合、地点Aから地点Bに向かっ 
て光を放ち、地点Bに光が到達するのにかかった時間を測定できれば、その距離がわかります。 
この方法により、他の方法では測定が困難であったりほとんど不可能であるような場所・
空間の距離測定が可能となります。例えば、大地の断層の変動幅や山の高さの変位などをmm 
の精度で、また、地球と月との距離(約38万 [km])を1 [m]以下の精度で求めることがで 
きます。測定の手法としては、地点Aに設置された光源装置から地点Bに向かってレーザー 
光を照射し、地点Bで反射された光を地点Aに置いた受光器で受け、発振時と受光時の時 
間差を測定するというものがあります。
その時間差を$\Delta t$ [s]とするとAB間の距離$L$ [m]は、光 
速$c$ [m/s]を用いて、
\[
L=\frac{1}{2}\Delta t \cdot c
\]
となる。

この方法はより日常的な距離の測定にも利用できます。今回の実験では50 [cm]$\sim$数 [m] 
の測定を試みて、レーザー光による距離測定の原理を理解することを目的とします。
使用す るレーザー光は、半導体レーザーで赤色の可視光です(波長:670 [nm])。
このレーザー 光を搬送波として周波数約30 [MHz]〜60 [MHz]の信号波を乗せた
変調波が発光装置から発振され、その変調波を用いて測定を行います。






\newpage

\jikken

\begin{itemsquarebox}[c]{\bf 実験用具}
発光装置、受光装置、反射鏡、2現象オシロスコープ、電子計数装置
\end{itemsquarebox}

\bigskip

\subjikken{オシロスコープによる波の周期測定}

\begin{enumerate}

\item リード線で、発光装置とオシロスコープのアース端子を接続します。

\item 電子計数装置のTTL入力端子と発光装置の変調周波数計測用端子を接続します。

\item 2現象オシロスコープのCh.1と発光装置の変調波計測用出力端子を同軸ケーブル 
で接続します。その後、接続したすべての装置の電源をONにします。

\item 波形がうまくモニタできるようにオシロスコープの垂直軸と時間軸の調整を行っ 
た後、CURSOR MODEボタンを用いて変調波の周期と周波数を読み取ります。

\item  発光装置横のつまみを回して、電子計数装置で周波数を確認しながら、変調波の
周波数を30 [MHz]、40 [MHz]、50 [MHz]と変化させてみます。(この時、電子計
数装置に表示されている周波数は1/1000に分周しているので実際の値は、$\times 1000$
になります。)それぞれの周波数に対して\maru{4}の操作を繰り返し、周波数と周期の 
測定を行います。

\end{enumerate}

\subjikken{レーザー光による距離測定 }

\begin{enumerate}

\item 発光装置および受光装置を地点Aに、反射鏡を地点Bに設置します。

\item リード線で、発光装置と受光装置のそれぞれのアース端子を接続します。さらに 
オシロスコープのアース端子とも接続します。

\item 電子計数装置のTTL入力端子と発光装置の変調周波数計測用端子を接続します。

\item 2現象オシロスコープのCh.1と発光装置の変調波計測用出力端子および2現象オ 
シロスコープのCh.2と受光装置の変調波計測用出力端子をそれぞれ同軸ケーブ 
ルで接続します。その後、接続したすべての装置の電源をONにします。

\item  まず測定の誤差を減らす為の操作として、発光装置と受光装置を10 [cm]程度離し 
て向かい合わせに置きオシロスコープに2つの波形を映し出します。(レーザー光 
が受光素子に当たっていればオシロスコープのCh.2の波形がモニタできます。) 
Ch.1、Ch.2の波形の高さが、等しくなるようにオシロスコープの垂直軸の調整 
を行った後、CURSOR MODEボタンを用いて2個の波形の位相差を$\Delta t_0$ [ns]と 
して読み取り記録します。

\item ここからは本測定に入ります。発光装置のレンズを回し、レーザー光の焦点が受 
光装置側で3 [mm]程度の円状になるように調整します。

\item 受光装置へのレーザー光の反射光は、受光装置の受光窓内部にある受光素子の中 
心に当たるように、反射鏡の微調節ねじを使い調整します。レーザー光が受光素 
子に当たっていればオシロスコープのCh.2の波形がモニタできます。

\item Ch.1、Ch.2の波形の高さが、等しくなるようにオシロスコープの垂直軸の調整を
行った後、CURSOR MODEボタンを用いて2個の波形の位相差を$\Delta t$ [ns]として
読み取り記録します。同時に電子計数装置に表示されている値も記録します。

\item AB間の距離を
\[
L=\frac{1}{2}(\Delta t -\Delta t_0)\cdot c
\]
を使って計算し、記録します。

\item 参考値として、AB間の距離を巻尺で測定します。測定値と参考値との間に誤差が 
生じた場合、それについて比較・考察しましょう。

\item 地点Bを変えて、\maru{6}以降を4回繰り返します。

\end{enumerate}

\paragraph{注意}
レーザー光源部、およびミラーの反射光は、失明の危険があるので、絶対にの
ぞき込まないようにして下さい。
受光素子にレーザー光が当たっているかどうか確認するには、受光波計用出力 
端子の出力をオシロスコープでモニタし、光路の遮蔽・透過で信号を確認して 
下さい。


