%!TEX root = ../main.tex
%
% 電子の比電荷の測定
% レポート用紙
%

\stepcounter{section}
\section*{電子の比電荷の測定}

\begin{center}
\begin{tabular}{|c|c|c|c|}
\hline
\parbox[c][1.2cm][c]{0cm}{}学籍番号 & \hspace{3cm} & 名前 & \hspace{6cm} \\
\hline
\parbox[c][1.2cm][c]{0cm}{}実験日時 & \multicolumn{3}{|l|}{   年  月  日  曜日  時限}\\
\hline
\parbox[c][2.0cm][c]{0cm}{}共同実験者 & \multicolumn{3}{|l|}{}\\
\hline
\end{tabular}
\end{center}

\subsection*{実験の目的と原理}

\newpage

\subsection*{測定値および計算}

\subjikken{}


\subsubsection*{機器の測定前準備チェックリスト}

\begin{tabular}{|r|p{14cm}|p{0.7cm}|}
\hline
1 & 電源装置のスイッチがOFFであることを確認 & \\
\hline
2 & $e/m$測定実験装置の電流調整つまみ:最小 & \\
\hline
3 & $e/m$測定実験装置の焦点調整:マーク真上& \\
\hline
4 & $e/m$測定実験装置のスイッチ:$e/m$測定 & \\
\hline
5 &  真空管用電源装置つまみ:すべて最小(電圧調整つまみは300V) & \\
\hline
6 & 真空管用電源装置スタンバイスイッチ:待機 & \\
\hline
7 & 直流電源装置\underline{\bf 電流}調整つまみ:両方とも最大 & \\
\hline
8 & 直流電源装置\underline{\bf 電圧}調整つまみ:粗調整最小、微調整真ん中 &\\
\hline
9 & 真空管用電源装置A電源とヒータを配線:\underline{6.3V}(B/C電源では無い!!) & \\
\hline
10 & 真空管用電源装置B電源と電極を配線 & \\
\hline
11 & 直流電源装置、電流計(5A)とヘルムホルツコイルを配線 & \\ 
\hline
12 & 電圧計(300V)を接続 & \\
\hline
13 & \underline{教員による配線のチェック} &\\
\hline\hline
14 & 電源に接続し、機器のスイッチをいれる &\\
\hline
15 & 真空管用電源装置スタンバイスイッチ:出力 & \\ 
\hline
16 & 真空管用電源装置B電源から150〜300Vの電圧を出力  & \\
\hline
17 & ヘルムホルツコイルの電流を調整して円を描くようにする  & \\
\hline
18 & 焦点調整でできるだけ電子ビームを細くする  & \\
\hline
19 & 加速電圧、ヘルムホルツコイルの電流、軌道半径を測定し記録する  & \\
\hline
\end{tabular}


\subsubsection*{測定値と計算結果}

\hspace*{-\parindent}
\begin{tabular}{|r|c|c|c|c|}
\hline
No & 加速電圧 $V$ [V] & コイルの電流 $I$ [A] & 軌道半径 $r$ [m] & 比電荷$e/m_e$の値 [C/kg] \\
\hline\hline
1&&&&\\
\hline
2&&&&\\
\hline
3&&&&\\
\hline
4&&&&\\
\hline
5&&&&\\
\hline
6&&&&\\
\hline
7&&&&\\
\hline
8&&&&\\
\hline
\end{tabular}

\subsubsection*{電子の比電荷の平均値}

\[
\frac{e}{m_e} = \hspace{5cm} \text{[C/kg]}
\]

\subsubsection*{電子の質量の平均値}

\[
m_e = \hspace{5cm} \text{[kg]}
\]


\subsection*{考察}
\begin{itemize}
\item 電荷や電子の質量を直接測定するのではなく、比電荷を測定する理由は何でしょうか?
\end{itemize}

