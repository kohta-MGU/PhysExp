%!TEX root = ../reports2.tex
%
% 気柱の固有振動と音速の測定
% レポート用紙
%

\stepcounter{section}
\section*{気柱の固有振動と音速の測定}

\begin{center}
\begin{tabular}{|c|c|c|c|}
\hline
\parbox[c][1.2cm][c]{0cm}{}学籍番号 & \hspace{3cm} & 名前 & \hspace{6cm} \\
\hline
\parbox[c][1.2cm][c]{0cm}{}実験日時 & \multicolumn{3}{|l|}{   年  月  日  曜日  時限}\\
\hline
\parbox[c][2.0cm][c]{0cm}{}共同実験者 & \multicolumn{3}{|l|}{}\\
\hline
\end{tabular}
\end{center}

\subsection*{実験の目的}

\vspace{5cm}


\subsection*{固有振動の観察と測定}
\subjikken{音速の測定(気柱の長さ固定)}
\subsubsection*{各固有振動状態の様子(スケッチ)}
実験時の室温
\[
t=\qquad\qquad\qquad [{}^\circ \text{C}]
\]

\newpage

\subsubsection*{気温から求めた音速}
\[
v=\qquad\qquad\qquad \text{[m/s]}
\]


\subsubsection*{各固有振動状態の測定}

\hspace*{-\parindent}
\begin{tabular}{|p{2.0cm}|p{3.2cm}|p{3.2cm}|p{3.2cm}|}
\hline
固有振動 ($n$倍振動)
 & 発振器の振動数 $\nu_n$ [Hz] & 半波長 $\lambda_n/2$ [m] & 音速 $v$ [m/s]  \\
\hline\hline
$n=3$&&&\\
\hline
$n=5$&&&\\
\hline
$n=7$&&&\\
\hline
$n=9$&&&\\
\hline
$n=11$&&&\\
\hline
$n=13$&&&\\
\hline
\end{tabular}

\vspace{2cm}

\subjikken{音速の測定(振動数固定)}



\subsubsection*{各固有振動状態の測定}


\hspace*{-\parindent}
\begin{tabular}{|p{2.0cm}|p{3.2cm}|p{4.5cm}|p{3.2cm}|}
\hline
固有振動 ($n$倍振動)
 & 気柱の長さ $L_n$ [m] & 波長 $\lambda$ [m] & 音速 $v$ [m/s]  \\
\hline\hline
$n=3$&&\diagbox[dir=SW,width=4.93cm,height=1.09\line,trim=r]{}{}&\diagbox[dir=SW,width=3.63cm,height=1.09\line,trim=r]{}{}\\
\hline
$n=5$&&$2(L_5-L_3)=$&\\
\hline
$n=7$&&$2(L_7-L_5)=$&\\
\hline
$n=9$&&$2(L_9-L_7)=$&\\
\hline
$n=11$&&$2(L_{11}-L_9)=$&\\
\hline
$n=13$&&$2(L_{13}-L_{11})=$&\\
\hline
\end{tabular}


\newpage




\subsection*{考察}

\begin{itemize}

\item 固有振動から求めた音速と気温から求めた音速を比較し、その差(誤差)について考察しましょう。

\vspace{6cm}

\item 両方の端が開いた管(開管)について、基本振動、2倍振動、3倍振動の模式図を描き、半波長に相当する部分を
図中に示してみましょう。

\vspace{6cm}

\item 気柱の固有振動を応用した身の回りのものについて、いろいろな例を調べてみましょう。

\end{itemize}

