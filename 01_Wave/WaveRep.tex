%!TEX root = ../main.tex
%
% 波の性質
% レポート用紙
%

\stepcounter{section}
\section*{波の性質}

\begin{center}
\begin{tabular}{|c|c|c|c|}
\hline
\parbox[c][1.2cm][c]{0cm}{}学籍番号 & \hspace{3cm} & 名前 & \hspace{6cm} \\
\hline
\parbox[c][1.2cm][c]{0cm}{}実験日時 & \multicolumn{3}{|l|}{   年  月  日  曜日  時限}\\
\hline
\parbox[c][2.0cm][c]{0cm}{}共同実験者 & \multicolumn{3}{|l|}{}\\
\hline
\end{tabular}
\end{center}

\subsection*{実験の目的}

\vspace{5cm}


\subsection*{観察}
\subjikken{}
\subsubsection*{スリットを通った水波の様子(スケッチ)}
\newpage

\subsubsection*{波源を動かした時の水波の様子(スケッチ)}
\vspace{6.5cm}

\subsubsection*{波源を動かす速度を変えた場合(スケッチ)}
\vspace{6.5cm}

\subsubsection*{2つの波源による水波の干渉(スケッチ)}


\newpage




\subsection*{考察}

\begin{itemize}

%\item 水の深さが変化した時の波の伝わり方の違いについて考察しましょう。
%
%\vspace{6cm}

\item 衝撃波が発生する理由と条件について考察しましょう。

\vspace{6cm}

\item 今回実験した波の性質は我々の身の回りでどのような現象として見られるでしょうか? 色々な例を調べてみましょう。

\end{itemize}

